\documentclass[12pt]{article}

\usepackage{amsmath}
\usepackage[margin = 1in]{geometry}
\usepackage{graphicx}
\usepackage{booktabs}
\usepackage{natbib}

\usepackage{lipsum}
\usepackage[colorlinks=true, citecolor=blue]{hyperref}



\title{Proposal: What is the impact of weather catastrophe frequency and severity on property insurance policy 
prices, and how can homeowners alleviate financial strain?}
\author{Carol Li\\
  University of Connecticut
}

\begin{document}
\maketitle


\paragraph{Introduction}
Weather catastrophes, including hurricanes, tornados, wildfires, hail storms, and other extreme weather events, have been increasingly 
prevalent and severe in recent years. These events pose significant risks to homeowners and carry substantial financial implications. 
This research proposal aims to investigate how the frequency and severity of weather catastrophes impact the pricing of property 
insurance policies and to identify strategies that can be developed to mitigate the financial burden on homeowners. Previous studies 
in the field \cite[e.g.,][]{kraehnert2021} have provided valuable insights, and we seek to build upon their findings. 

\paragraph{Specific Aims}
Our study focuses on three main objectives pertaining to weather catastrophes and property insurance pricing. To start with, we seek 
to explore the overarching research query: ``What is the impact of weather catastrophe frequency and severity on property insurance policy 
prices, and how can homeowners alleviate financial strain?`` This fundamental inquiry will serve as the basis for our investigation.

To examine this question thoroughly, we intend to analyze how variations in insured loss over time correlate with changes in both the 
frequency and severity of weather catastrophes such as hurricanes, tornados, wildfires, and hail storms. By studying these factors' 
influence on policy pricing fluctuations, we aim to gain a comprehensive understanding of their interplay.

Moreover, we plan to convert this primary research question into specific statistical inquiries that employ data science techniques. 
These focused queries include evaluating whether there exists a correlation between weather catastrophe occurrence rates and 
alterations in insured loss over time. Additionally, through data-driven analysis methods, we aim to identify effective strategies 
for alleviating homeowners' financial burdens caused by these catastrophic events.

By successfully achieving these aims through quantitative methodologies rooted in data science principles, our research endeavors 
hold promise for providing valuable insights that can inform practices within the insurance industry. Finally, this knowledge has 
potential implications for public policy decisions aimed at enhancing disaster resilience amidst escalating climate-related risks.


\paragraph{Data}
We will primarily rely on information acquired from reputable sources for our research, particularly the Insurance Information 
Institute (III) \cite{iii}, which has regularly offered thorough information on catastrophic occurrences and its consequences. We will also 
depend on Aon's 2023 Weather, Climate, and Catastrophe Insight Report \cite{aon}, which acts as a trustworthy and up-to-date reference for data 
regarding catastrophic events. This report delivers in-depth insights into the frequency, severity, and financial effects of numerous 
weather-related catastrophes, including hurricanes, wildfires, severe storms, and more. The III gathers information on the frequency 
and severity of catastrophic events, the insured losses incurred, and their effects on the insurance industry, using Aon as one of 
its main references. Data on current trends and patterns in weather catastrophes and their economic effects can be found in Aon's 
2023 report. Additionally, information on insured and uninsured losses, claim histories, and regional effects can all be found in 
this report which are crucial to our analysis. The report includes a variety of factors, such as the frequency of weather disasters, 
their geographic distribution, the financial losses connected to each incident, and the degree to which these losses are insured. 
In addition, the report provides information on demographics, policy coverage specifics, and insurance premiums, enabling us to 
investigate the connection between weather catastrophes and home insurance premiums.

\paragraph{Research Design and Methods}
In order to fully address the research objectives, our research design and methods will take an integrated approach. We will start by 
performing a thorough analysis of the information obtained from the Insurance Information Institute (III) \cite{iii}and Aon's 2023 Weather, 
Climate, and Catastrophe Insight Report\cite{aon}. To comprehend the relationships between the frequency and severity of weather catastrophes 
and the pricing of property insurance, this analysis will use statistical techniques like regression analysis and correlation studies. 

On the basis of historical weather catastrophe data and other relevant variables, we will use predictive modeling techniques to 
forecast changes in the pricing of property insurance premiums in order to address the data science aspects of our research. 
Additionally, we'll pinpoint the elements most responsible for changes in insurance costs in the face of frequent and severe weather 
catastrophes.

Our research will be executed over a well-defined timeline, which includes data collection, statistical analysis, predictive modeling, 
and the synthesis of findings into actionable recommendations. This comprehensive approach will ensure that we address both the 
statistical and data science aspects of our research question and provide a better understanding of the issue at hand.

\paragraph{Discussion}
In order to comprehend the effects of weather catastrophes on house insurance price and propose mitigation techniques, our research 
faces a number of difficulties. The intricacy of the dataset obtained from the Insurance Information Institute and Aon's 2023 Weather, 
Climate, and Catastrophe Insight Report presents a substantial problem. These datasets need careful data preprocessing, cleansing, and 
validation because they are packed with information. Additionally, because it requires negotiating complex relationships among the data
, comprehending the outcomes of statistical analysis and predictive modeling can be challenging.

Furthermore, the accuracy and availability of the data may provide challenges for our research. Even if Aon's report and III's 
statistics are reliable sources, they might have gaps in their geographic scope, breadth, or level of detail for some factors. 
These restrictions might affect how thorough our findings are.

We have developed a fallback plan to deal with these difficulties and constraints. We shall give data quality and analytical 
robustness the highest priority in the event that unanticipated problems occur. To ensure the validity of our findings, we will 
carefully record our data preprocessing procedures. In addition, if data restrictions occur, we will be upfront about them in our 
final report and indicate how they might affect our results.

\paragraph{Conclusion}
In conclusion, our study proposal describes a thorough and multi-methodological approach to investigate the effect of weather 
catastrophes on the pricing of home insurance policies and the development of methods to lessen the financial burden on homeowners. 
Our research attempts to provide important insights into this urgent topic by utilizing data from reliable sources including the 
Insurance Information Institute and Aon's 2023 Weather, Climate, and Catastrophe Insight Report. We will be able to present a 
comprehensive viewpoint on the topic and contribute to informed policy decisions and industrial practices thanks to the mix of 
statistical analysis, and data science approaches.

\bibliography{refs}
\bibliographystyle{chicago}

\end{document}