\documentclass[12pt]{article}

\usepackage{amsmath}
\usepackage[margin = 1in]{geometry}
\usepackage{graphicx}
\usepackage{booktabs}
\usepackage{natbib}

\usepackage{lipsum}
\usepackage[colorlinks=true, citecolor=blue]{hyperref}



\title{Proposal: Something Interesting}
\author{Carol Li\\
  University of Connecticut
}

\begin{document}
\maketitle


\paragraph{Introduction}
Weather catastrophes, including hurricanes, tornados, wildfires, hail storms, and other extreme weather events, have been increasingly 
prevalent and severe in recent years. These events pose significant risks to homeowners and carry substantial financial implications. 
This research proposal aims to investigate how the frequency and severity of weather catastrophes impact the pricing of property 
insurance policies and to identify strategies that can be developed to mitigate the financial burden on homeowners. Previous studies 
in the field \citep[e.g.,][]{kraehnert2021} have provided valuable insights, and we seek to build upon their findings. 

\paragraph{Specific Aims}
Our study focuses on three main objectives pertaining to weather catastrophes and property insurance pricing. To start with, we seek 
to explore the overarching research query: "What is the impact of weather catastrophe frequency and severity on home insurance policy 
prices, and how can homeowners alleviate financial strain?" This fundamental inquiry will serve as the basis for our investigation.

To examine this question thoroughly, we intend to analyze how variations in insured loss over time correlate with changes in both the 
frequency and severity of weather catastrophes such as hurricanes, tornados, wildfires, and hail storms. By studying these factors' 
influence on policy pricing fluctuations, we aim to gain a comprehensive understanding of their interplay.

Moreover, we plan to convert this primary research question into specific statistical inquiries that employ data science techniques. 
These focused queries include evaluating whether there exists a correlation between weather catastrophe occurrence rates and 
alterations in insured loss over time. Additionally, through data-driven analysis methods, we aim to identify effective strategies 
for alleviating homeowners' financial burdens caused by these catastrophic events.

By successfully achieving these aims through quantitative methodologies rooted in data science principles, our research endeavors 
hold promise for providing valuable insights that can inform practices within the insurance industry. Finally, this knowledge has 
potential implications for public policy decisions aimed at enhancing disaster resilience amidst escalating climate-related risks.


\lipsum[2]

\paragraph{Data}
Hopefully, you have identified the data needed for your project. Give a
description about it.

\lipsum[3]

\paragraph{Research Design and Methods}
What design or methods will you use?
Cite relevant references~\citep[e.g.,][]{}.

\lipsum[4]

\paragraph{Discussion}
What are the most challenge parts of the task?
What are the limitations of your work? What is your fall-back plan if
something unexpected happens?

\lipsum[5]

\bibliography{../manuscript/refs}
\bibliographystyle{chicago}

\end{document}