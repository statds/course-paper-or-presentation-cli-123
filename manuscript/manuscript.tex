\documentclass[12pt]{article}

%% preamble: Keep it clean; only include those you need
\usepackage{amsmath}
\usepackage[margin = 1in]{geometry}
\usepackage{graphicx}
\usepackage{booktabs}
\usepackage{natbib}

% highlighting hyper links
\usepackage[colorlinks=true, citecolor=blue]{hyperref}


\title{What is the impact of weather catastrophe frequency and severity on property insurance policy 
prices, and how can homeowners alleviate financial strain?}
\author{Carol Li\\
    University of Connecticut
}

\begin{document}
\maketitle

\begin{abstract}
Weather catastrophes, like hurricanes, wildfires, hail storms, etc., have become more common and severe in recent years, which poses 
significant risks to homeowners and has significant financial consequences. The purpose of this research paper is to look into the 
impact of weather catastrophe frequency and severity on property insurance policy prices, and also, to identify strategies for easing 
the financial burden on homeowners. This research combines statistical analysis and data science techniques, drawing on data from 
reliable sources such as the Insurance Information Institute and Aon's 2023 Weather, Climate, and Catastrophe Insight Report. The 
research aims to provide helpful insights into the relationships between the frequency and severity of weather catastrophes and 
property insurance pricing in order to inform insurance industry practices and public policy decisions to improve resilience to 
catastrophes.
\end{abstract}


\section{Introduction}
\label{sec:intro}
In recent years, weather catastrophes have become more frequent and severe, which poses a potential threat to homeowners' safety as 
well as financial impacts. As the frequency and severity of disasters increase, the significance of the ability of homeowners to 
protect their homes using property insurance also increases. This research will focus on exploring the relationship between the 
frequency and severity of weather catastrophes and the pricing of property insurance.

Climate disasters have become more frequent and severe in recent years. These incidents not only put homeowners' safety at risk but 
also have significant economic consequences. As weather-related risks increase in frequency and severity, the ability of homeowners to 
protect their valuable assets through property insurance has become increasingly important This paper aims to explore the relationship 
between the frequency and severity of weather hazards and property insurance policies and prices

The importance of this study is the fact that weather disasters carry profound amount of consequences, affecting individual homeowners 
and the insurance industry as a whole. Homeowners are the ones who typically bear the financial burden of disasters, and face 
insurance premiums, deductibles, and coverage limitations at the same time. On the other hand, the insurance company is tasked to 
manage the increased risks associated with these events, which can have a significant impact on their financial stability.


While previous research has examined different aspects of this complex issue, including the impact of weather risks on insurance 
prices, there is much more waiting to be discovered. Our research is based on valuable insights into the value of previous work 
\cite{hurricaneco} aimed at assets that have provided insurance pricing of weather frequency and severity. It provides a general 
understanding of how frequency and severity affect the pricing of property insurance. We are also trying to identify effective 
strategies that can reduce the financial burden of homeowners in the face of these major disasters.


This research has used several methods, including statistical analysis and data science methods. To achieve our goals, we will draw 
data from reliable sources, such as the Insurance Information Agency and Aon's 2023 Weather, Climate and Disaster Investigation Report. 
This study is not only timely but critically important at a time of increasing weather-related risks, with the goal of contributing to 
homeowner well-being and the stability of the property insurance market

The rest of the paper is organized as follows.
The data will be presented in Section~\ref{sec:data}.
The methods are described in Section~\ref{sec:meth}.
The results are reported in Section~\ref{sec:resu}.
A discussion concludes in Section~\ref{sec:disc}.


\section{Data}
\label{sec:data}



\section{Methods}
\label{sec:meth}




\section{Results}
\label{sec:resu}




\section{Discussion}
\label{sec:disc}



\bibliography{refs}
\bibliographystyle{mcap}

\end{document}