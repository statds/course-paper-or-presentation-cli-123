\documentclass[12pt]{article}

%% preamble: Keep it clean; only include those you need
\usepackage{amsmath}
\usepackage[margin = 1in]{geometry}
\usepackage{graphicx}
\usepackage{booktabs}
\usepackage[numbers]{natbib}
\usepackage{amsfonts}
\usepackage{dingbat}


% highlighting hyper links
\usepackage[colorlinks=true, citecolor=blue]{hyperref}

% double spacing
\usepackage{setspace}
\doublespacing

\title{The Impact of Extreme Weather Events on Property Insurance Pricing}
\author{Carol Li\\
    University of Connecticut
}

\begin{document}
\maketitle

\begin{abstract}
Weather catastrophes, like hurricanes, wildfires, hail storms, etc., have become more common and severe in recent years, which poses 
significant risks to homeowners and has significant financial consequences. The purpose of this research paper is to look into the 
impact of weather catastrophe frequency and severity on property insurance policy prices, and also, to identify strategies for easing 
the financial burden on homeowners. This research combines statistical analysis and data science techniques, drawing on data from 
reliable sources such as the Insurance Information Institute and Aon's 2023 Weather, Climate, and Catastrophe Insight Report. The 
research aims to provide helpful insights into the relationships between the frequency and severity of weather catastrophes and 
property insurance pricing in order to inform insurance industry practices and public policy decisions to improve resilience to 
catastrophes.
\end{abstract}


\section{Introduction}
\label{sec:intro}
The frequency and severity of weather disasters have increased in recent years, posing threats to homeowners' safety and finances. 
This study investigates the relationship between these disasters and property insurance pricing. As climate disasters become more 
severe, homeowners face increased risks, necessitating a closer look at their ability to protect assets through insurance.

The study emphasizes the catastrophic impact of natural disasters on homeowners and the insurance industry. The financial burden 
falls on homeowners, who must manage insurance premiums, deductibles, and coverage limitations. Likewise, insurance companies face 
increased risks, threatening their financial stability.

While previous research has touched on weather risks and insurance prices, our study delves deeper into the dynamics of frequency and 
severity in property insurance pricing, based on valuable insights \cite{hurricaneeco}. We want to find effective ways to reduce the financial burden 
on homeowners during major disasters.

We draw data from reliable sources such as the Insurance Information Agency, National Insurance Centers for Environmental Information, 
and Aon's 2023 Weather, Climate, and Disaster Investigation Report using statistical analysis and data science. This study aims to 
improve homeowner well-being and the stability of the property insurance market in the face of rising weather-related risks.


The rest of the paper is organized as follows.
The data will be presented in Section~\ref{sec:data}.
The methods are described in Section~\ref{sec:meth}.
The results are reported in Section~\ref{sec:resu}.
A discussion concludes in Section~\ref{sec:disc}.


\section{Data}
\label{sec:data}
To comprehend the impact of weather-related catastrophic events on property insurance pricing, we will begin by examining a dataset of 
billion-dollar disasters. This dataset provides crucial information on event frequency, financial cost, and other key parameters. Our 
data sources include the National Centers for Environmental Information (NCEI)\cite{ncei}, the Insurance Information Institute 
(III)\cite{iii}, Aon\cite{aon} and the National Association of Insurance Commissioners (NAIC)\cite{naic}, all are recognized for their 
reliability in documenting catastrophic events and their associated economic and insured losses.

These data sources, sourced from leading insurance analytics firms and research organizations, provide an important foundation and 
context for assessing the relationship between intensifying extreme weather events and rising property insurance costs. Using decades 
of specialized knowledge to track climate impacts and pricing trends across vulnerable geographies.

Aon is at the forefront of risk analysis in the insurance industry, with dedicated catastrophe model development and annual global 
climate reports tracking disaster losses. This establishes Aon as an unrivaled resource for linking increasing weather perils to 
property insurance pricing trends.

The Insurance Information Institute (III), a trusted insurance trade group, provides proprietary research and public education on aligning 
property insurance costs with escalating climate risks confronting communities across risk-prone regions.

The National Centers for Environmental Information (NCEI) provides data backbone to guide evidence-based assessment of how record-breaking 
weather extremes are translating to property insurance market dynamics by maintaining the nation's authoritative disaster cost 
database.

The National Association of Insurance Commissioners (NAIC) facilitates transparency from insurers on pricing and product shifts caused by 
worsening climate factors such as hurricanes, flooding, and wildfires through regulatory oversight across states, structured data 
calls, and cross-collaborations.

Our primary dataset, obtained from the National Centers for Environmental Information (NCEI)\cite{ncei}, spans between the years from 
1980 to 2023. We have organized the data into various temporal segments to facilitate analysis. Table \ref{tab:bil_dol_disasters} 
summarizes key statistics from this dataset, which provides insights into the frequency of catastrophic events, total financial cost, and 
associated fatalities. These temporal segments, ranging from the 1980s to the present day, enable us to assess trends over time and 
help to identify potential patterns.

\begin{table}[h]
    \caption{NEIC: Billion-Dollar Disasters Data}
    \label{tab:bil_dol_disasters} 
    \centering
    \resizebox{\columnwidth}{!}{\begin{tabular}{|l|c|c|c|c|c|c|c|c|}
        \hline
        Time Period & Billion-Dollar Disasters & Events/Year & Cost & Percent of Total Cost & Cost/Year & Deaths & Deaths/Year \\
        \hline
        1980s (1980-1989) & 33 & 3.3 & \$213.6B & 8.1\% & \$21.4B & 2,994 & 299 \\
        1990s (1990-1999) & 57 & 5.7 & \$326.8B & 12.4\% & \$32.7B & 3,075 & 308 \\
        2000s (2000-2009) & 67 & 6.7 & \$604.2B & 22.9\% & \$60.4B & 3,102 & 310 \\
        2010s (2010-2019) & 131 & 13.1 & \$967.4B & 36.7\% & \$96.7B & 5,227 & 523 \\
        Last 5 Years (2018-2022) & 90 & 18.0 & \$623.0B & 23.6\% & \$124.6B & 1,751 & 350 \\
        Last 3 Years (2020-2022) & 60 & 20.0 & \$456.0B & 17.3\% & \$152.0B & 1,460 & 487 \\
        Last Year (2022) & 18 & 18.0 & \$178.8B & 6.8\% & \$178.8B & 474 & 474 \\
        All Years (1980-2023)* & 372 & 8.5 & \$2,635.1B & 100.0\% & \$59.9B & 16,231 & 369 \\
        \hline
    \end{tabular}}
    \cite{ncei}
\end{table}



To further examine the financial implications of natural disasters, we turn to the Insurance Information Institute (III)\cite{iii}. 
Table \ref{tab:natural_cat_losses} for the year 2022 provides a breakdown of catastrophic events by peril, including data on the 
number of events, fatalities, economic losses, and insured losses. This granular information is essential for understanding the 
varying impacts of different types of natural catastrophes and their associated costs.

\begin{table}[h]
    \caption{Natural Catastrophe Losses in the United States by Peril, 2022 (in \$ millions)}
    \label{tab:natural_cat_losses}
    \centering
    \resizebox{\columnwidth}{!}{\begin{tabular}{|l|c|c|c|c|}
        \hline
        Peril & Number of Events & Fatalities & Economic Losses (2) & Insured Losses (3) \\
        \hline
        Tropical Cyclone & 3 & 157 & \$ 96,097 & \$ 53,203 \\
        Severe Convective Storm & 62 & 49 & \$ 37,232 & \$ 29,306 \\
        Wildfire, Drought, Heatwave & 26 & 65 & \$ 18,093 & \$ 8,902 \\
        Winter Storm & 13 & 123 & \$ 6,223 & \$ 4,128 \\
        Flooding & 15 & 72 & \$ 7,234 & \$ 3,346 \\
        Total & 119 & 466 & \$ 164,879 & \$ 98,885 \\
        \hline
    \end{tabular}}
    \cite{iii}
\end{table}

This chart \ref{fig:disaster_freq} shows the percentage frequency of years from 1980-2023 that experienced different numbers of billion-dollar disasters. It 
looks at years with 1 or more, up to 5 or more billion-dollar disaster events. The chart helps to characterize the concentration and 
frequency of multiple disasters in a single year.

\begin{figure}[ht]
    \centering
    \includegraphics[width=0.8\linewidth]{NCEI US disaster freq.pdf}
    \label{fig:disaster_freq}
    \cite{ncei}
\end{figure}

The Insurance Information Institute (III) \cite{iii} also supplies data on insured property losses in the United States for the years 
2013-2022. Table \ref{tab:insured_prop_losses} presents both the nominal loss values when the events occurred and their equivalent 
values in 2022 dollars. Analyzing this information will allow us to assess how insured losses have evolved over the past decade.


\begin{table}[h]
    \caption{Estimated Insured Property Losses, U.S. Natural Catastrophes, 2013-2022 (in \$ billions)}
    \label{tab:insured_prop_losses}
    \centering
    \begin{tabular}{|c|c|c|}
        \hline
        Year & In dollars when occurred & In 2022 dollars (2) \\
        \hline
        2013 & \$ 24.1 & \$ 31.0 \\
        2014 & \$ 23.2 & \$ 29.2 \\
        2015 & \$ 22.9 & \$ 28.8 \\
        2016 & \$ 31.6 & \$ 39.3 \\
        2017 & \$ 130.9 & \$ 158.7 \\
        2018 & \$ 60.4 & \$ 71.6 \\
        2019 & \$ 38.8 & \$ 45.2 \\
        2020 & \$ 81.0 & \$ 93.3 \\
        2021 & \$ 93.3 & \$ 102.7 \\
        2022 & \$ 98.8 & \$ 99.9 \\
        \hline
    \end{tabular}
    \cite{iii}
\end{table}    
  

This figure \ref{fig:disaster_threats} shows the cumulative billion-dollar disaster costs in the United States from 1980-2023 on a year-to-date basis. It 
illustrates how annual disaster costs accumulate through the year, with key major disaster event years highlighted. The table 
provides context on total disaster costs over time.

\begin{figure}[ht]
    \centering
    \label{fig:disaster_threats}
    \includegraphics[width=0.8\linewidth]{NAIC HO threats.pdf}
    \cite{naic}
\end{figure}

This figure \ref{fig:regional_disasters} shows the number of billion-dollar disaster events by type in the United States from 1980-2023. It breaks down event counts 
for droughts, flooding, freezes, severe storms, tropical cyclones, wildfires, and winter storms. The total disaster cost is also shown 
accumulated across years. The chart gives overview insight into disaster frequency and cost by peril.


\begin{figure}[ht]
    \centering
    \includegraphics[width=0.8\linewidth]{NAIC Property Threat by Regions.pdf}
    \label{fig:regional_disasters}
    \cite{naic}
\end{figure}

As evidenced by this chart \ref{fig:10economic_loss} of the ten most expensive global insured catastrophe losses from 1900 to 2022, 
the rising frequency and severity of extreme weather events in the last two decades has increased the risks and costs for property 
insurers. Hurricanes that have ravaged North America and the Caribbean in recent years have caused almost all of the disasters, with 
seven of the ten occurring in the last 15 years. The confluence of billion-dollar hurricane disasters, as well as rapidly rising 
nominal and inflation-adjusted insured losses, highlight the insurance industry's mounting climate-change costs. With three of the 
four largest loss events occurring in 2021-2022 alone, property insurers face escalating climate threats and exposures that 
necessitate significant adjustments to rates, coverages, and risk mitigation efforts in especially vulnerable coastal and island 
regions. Addressing these challenges through preventative resilience initiatives and innovative insurance solutions will be critical 
to enabling the industry to cover the property landscape of the future in a viable and sustainable manner.

\begin{figure}[ht]
    \centering
    \includegraphics[width=0.8\linewidth]{AON Top 10 Cyclones Economic Loss.pdf}
    \label{fig:10economic_loss}
    \cite{aon}
\end{figure}


This chart \ref{fig:10global_loss} of the top ten most economically destructive tropical cyclones in the world from 1900 to 2022 emphasizes the growing risks 
that hurricanes pose in a changing climate. With the United States being hit by all ten of the costliest storms since 2000, the trend 
toward repeated devastating hurricane seasons is clear and concerning for coastal communities. The clustering of five massive 
back-to-back hurricane events in the last six years, putting unprecedented strain on the US economy, lends credence to modeling that 
suggests tropical storms are becoming more destructive. Hurricane vulnerability and recovery costs will only increase as sea levels 
rise and waters warm, exposing more property than ever before.  This necessitates immediate action across the public and private 
sectors to strengthen pre-disaster mitigation and post-disaster resilience if the already record-breaking cost of hurricanes is not 
to skyrocket in the future. To address this escalating climate threat, innovative policy, infrastructure, community design, and 
financing tools must be combined.

\begin{figure}[ht]
    \centering
    \includegraphics[width=0.8\linewidth]{AON Top 10 Global Insured Loss.pdf}
    \label{fig:10global_loss}
    \cite{aon}
\end{figure}



\subsection{Key Equations}

To better understand the relationships between weather catastrophes, property insurance, and financial impact, we will first introduce 
several key equations that will act as a basis to guide our analysis:

\begin{equation}
    \label{eq:freq}
    \text{Frequency} = \frac{\text{Number of Loss Events}}{\text{Exposure Units}}
\end{equation}

The frequency equation \ref{eq:freq} calculates the frequency of loss events by dividing the number of loss events by exposure units.

\begin{equation}
    \label{eq:sev}
    \text{Severity} = \frac{\text{Total Loss Amount}}{\text{Number of Loss Events}}
\end{equation}

The severity equation \ref{eq:sev} is determined by dividing the total loss amount by the number of loss events.

\begin{equation}
    \label{eq:losscost}
    \text{Loss Cost} = \text{Frequency} \times \text{Exposure Units}
\end{equation}

The loss count equation \ref{eq:losscost} is the product of frequency and exposure units, providing insights into the overall loss count due to catastrophic events.

\begin{equation}
    \label{eq:premium}
    \text{Premium} = \lambda \cdot \text{Severity} \cdot \text{Exposure Units}
\end{equation}

The premium equation \ref{eq:premium} calculates the estimated premium based on the frequency, severity, and exposure units. This 
reflects the financial implications for homeowners and insurers.

We hope to gain a thorough understanding of the complex relationship between weather-related disasters and property insurance pricing 
by understanding these fundamental equations. These equations will serve as a basis for our data analysis, allowing us to effectively 
assess the impact of catastrophic events on property insurance while also assisting us in meeting our research objectives.

This comprehensive dataset, supported by granular data from the Insurance Information Institute \cite{iii}, serves as the foundation 
for our research. The various chronological segments, as well as the key equations, provide us with the resources that we need to 
delve deeper into the impact of weather-related catastrophes on property insurance, allowing us to effectively address our research 
objectives. 



\section{Methods}
\label{sec:meth}
We will estimate the influence of natural disasters on homeowners insurance premiums using panel regression models with state and year 
fixed effects:

\begin{equation} 
    \mathrm{Premium}_{ist} = \beta_0 + \beta_1 \cdot \mathrm{Disasters}_{ist} + \gamma_i + \delta_t + \epsilon_{ist}
\end{equation}

The dependent variable is the logged average premium in state $i$ and year $t$. The main independent variables are disaster measures 
for state-year:

\begin{itemize} 
    \item Number of events 
    \item Total cost 
    \item Insured losses (also known as loss cost\ref{eq:losscost})
    \item Indicators for peril types: hurricane, flood, severe storm, winter weather, wildfire, earthquake 
\end{itemize}


In order to isolate the relationships between disaster activity and average homeowners insurance premiums over time, we used 
panel regression techniques with state and year-fixed effects. To reduce omitted variable bias, models specifically control for 
time-invariant differences across states (via state dummies) as well as national trends (via year indicators). The premium measure is 
the dependent variable, with the key independent variables being disaster counts and cost totals.

These disaster measures capture state-specific exposure each year, providing granular, longitudinal pricing connectivity. Taking logs 
accounts for skewness and simplifies elasticity interpretation. We anticipate that worsening loss shocks will have a direct impact on 
insurers' underwriting decisions, resulting in measurable premium impacts. Differential sensitivity testing is possible by segmenting 
by peril type.

Given the skewed distribution of disaster cost variables, logs are used. Fixed effects by state and year account for time-dependent 
differences between states and national trends. We begin by looking at the impact of aggregated disaster activity. The models are then 
estimated by interacting the disaster measures with peril indicators in order to compare the sensitivity across event types. Quantile 
regression was also used to determine whether or not effects differ across the premium distribution.


\section{Results}
\label{sec:resu}
\subsection{Summary Statistics}
Table 4 \ref{tab:summary} presents summary statistics for the premium and disaster data. The disaster measures show substantial variability, 
highlighting the irregular nature of extremes. Hurricane and flooding perils account for the largest share of overall cost and 
insured losses. The regression results show that increasing natural disaster activity plays a statistically significant role in 
driving increases in homeowners insurance premiums across risk-exposed states. According to the elasticity estimates, premiums grow 
at a higher percentage rate than disaster costs.

\begin{table}[h]
    \label{tab:summary}
    \centering
    \begin{tabular}{|l|c|c|c|}
        \hline
        & Mean & SD & Range \\
        \hline
        Premium & \$\num{959.2} & \$\num{238.5208} & \$\num{536} - \$\num{1311} \\
        Number of Disasters & MeanDisaster & SDDisaster & MinDisaster - MaxDisaster \\
        Total Cost (\$) & MeanCost & SDCost & MinCost - MaxCost \\
        Insured Losses (\$) & MeanLosses & SDLosses & MinLosses - MaxLosses \\
        \hline
    \end{tabular}
    \caption{Summary Statistics}
    \cite{statista, ncei, FEMA}
\end{table}

\subsection{Regression Results}
Table 5 \ref{tab:reg_results} presents results from the panel regressions. The number of disasters and total cost are significantly 
associated with higher premiums based on the log-log specification. A 10\% increase in disasters corresponds to a 14.7\% rise in 
premiums. Meanwhile, a 10\% rise in total damage leads to a 23.1\% premium increase. In other words, 10\% increase in total catastrophe 
damages incurred in a state-year, for example, corresponds to a 23.1\% increase in annual average premiums. This emphasizes the 
exponentially rising underwriting risks and loss adjustments made by insurers based on compounding extreme event data.



\begin{table}[h]
    \caption{Regression Results}
    \label{tab:reg_results}
    \centering
    \begin{tabular}{|l|c|c|c|}
        \hline
        & (1) & (2) & (3) \\
        \hline
        Log(Premium) & 0.147$^{\ast}$ & & \\
        & (0.082) & & \\
        Log(Cost of Catastrophe Damages) & & 0.231$^{\ast}$ & 0.230$^{\ast}$ \\
        & & (0.115) & (0.115) \\
        Log(Insured Catastrophe Loss) & & & 0.147$^{\ast}$ \\    
        & & & (0.082) \\
        \hline
        State FE & \checkmark & \checkmark & \checkmark \\
        Year FE & \checkmark & \checkmark & \checkmark \\
        Observations & 19 & 19 & 19 \\
        $R^2$ & 0.160 & 0.193 & 0.160 \\
        \hline
    \end{tabular}
    
    \cite{statista, ncei}
  \end{table}
  

  The results presented in Table 6 \ref{tab:reg_peril} indicate that hurricane and flood disasters have the most 
  substantial impacts on insurance premiums. Specifically, a 10\% increase in hurricane or flood Damages are associated with 
  approximately a 0.01\% increase in premiums. Results also show that hurricanes and flooding disasters have a disproportionate 
  impact in comparison to other hazards. Climate change appears to be worsening exposure along the coast and in low-lying areas. 
  Additional findings indicate that winter storms may necessitate price changes in some states. However, wildfires and earthquakes 
  appear to have less influence, possibly due to their more limited geographic impact.

  


\begin{table}[h]
    \centering
    \caption{Disaster Effects by Peril}
    \label{tab:reg_peril}
    \begin{tabular}{|l|c|c|c|}
      \hline
      & (1) & (2) & (3) \\
      \hline
      Log(Severe Storm Cost) & 0.1776727 & & \\
      & & (0.05052779) & \\
      Log(Flood Cost) & & & \\
      & & & (0.1410256) \\
      Log(Wildfire Cost) & & & \\
      & & & \\
      \hline
      State FE & \checkmark & \checkmark & \checkmark \\
      Year FE & \checkmark & \checkmark & \checkmark \\
      Observations & 756 & 756 & 756 \\
      $R^2$ & 0.935 & 0.935 & 0.935 \\
      \hline
    \end{tabular}
    
    \cite{statista, ncei}
  \end{table}
  
  These panel regressions provide empirical evidence that worsening extremes are increasing premiums in line with rising expected loss 
  trends. Strategic risk-reduction policies will be critical in the future.



\section{Discussion}
\label{sec:disc}

The study found a statistically significant link between extreme weather events and homeowner insurance pricing. According to the 
results, more frequent and severe events are associated with higher premiums, especially for hurricanes and flooding perils. This is 
consistent with previous findings on the impact of catastrophic losses on insurance markets \cite{aon}.

However, there are limitations to the data collected. Analytics may benefit from having more granular, address-level premium data to 
better pinpoint local hazards. In addition, the model fails to take account of changes in exposures and vulnerabilities over time as risk 
factors. Further studies should be conducted to mitigate the influence of climate risk data change from detailed measures, which 
increases risks.

Still, these results demonstrate an interrelation between catastrophic incidents and rising property insurance premiums - two trends 
with many causes and effects that overlap significantly. As climate change intensifies weather extremes, homeowners could face 
financial strain - potentially undermining the stability of insurance markets and leading to economic instability for policy 
interventions like subsidized insurance policies\cite{iii}, resilience incentives, or residential investments\cite{kousky} that 
might provide relief. Climate risk modeling and risk-based pricing will become even more crucial to insurers as a means of providing 
long-term financial protection from rising extremes.




\bibliography{refs}
\bibliographystyle{chicago}

\end{document}